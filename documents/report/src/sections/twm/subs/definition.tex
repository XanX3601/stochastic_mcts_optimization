\subsection{Definition of the \acrshort{twm} problem}%
\label{sub:definition_of_the_twm_problem}

Our modelization is practically the same as the one defined by \citeauthor{comp_mcts_mo}\cite{comp_mcts_mo} with minor changes.
The \gls{twm} is set on a squared grid of cells \(X\).
Each cell \(x \in X\) is defined by two attributes:

\begin{itemize}
    \item \(B(x)\) is a boolean indicating if the cell \(x\) is on fire.
    \item \(F(x)\) is an integer representing the amount of fuel on cell \(x\). The more fuel, the longer it burns.
\end{itemize}

A state is then represented by the values of these two attributes for all \(x \in X\).
A state is considered final when no cell is burning anymore.

We have a set \(T\) of firefighter teams which we consider identical.
An action consists in placing each team \(t \in T\) setting the attribute \(a^{i} \in X\).
Every team can be assigned to any cell and can even be assigned to the same cells.

A burning cell consumes its fuel at a constant rate.
If a cell has no more fuel, it extinguishes.
The burning rate constitutes the time unite of the game. 
Meaning that, each turn, all burning cells consume one fuel, therefore:

\[
    F_{t+1}(x) = 
    \begin{cases}
        F_{t}(x)& \text{if } \lnot B_{t}(x) \lor F_{t}(x) = 0\\
        F_{t}(x) - 1& \text{otherwise }
    \end{cases}
\]
The evolution of the attribute \(B(x)\) is stochastic.
It is set on a terminal state machine.
This transition model is given by the following equation defining \(\rho_{1}\) and \(\rho_{2}\) and the figure~\ref{fig:B_state_machine}.

\[
    \rho_{1} = 
    \begin{cases}
        1 - \Pi_{y}(1 - P(x, y)B_{t}(x))& \text{if } F_{t}(x) > 0\\
        0& \text{otherwise }
    \end{cases}
\]
\[
    \rho_{2} = 
    \begin{cases}
        1& \text{if } F_{t}(x) = 0\\
        1 - \Pi_{i}(1 - S(x)\delta_{x}(a^{i}))& \text{otherwise }
    \end{cases}
\]

\begin{figure}[htpb]
    \ctikzfig{../../../figures/B_state_machine}
    \caption{\(B(x)\) transition model}%
    \label{fig:B_state_machine}
\end{figure}

\(P(x, y)\) represents the probability that \(y\) ignites \(x\) if \(y\) is burning.
We will only consider case where only neighbors cell can ignite each other.
\(S(x)\) is the probability that a firefighter team extinguish the cell \(x\).
\(\delta_{x}(a^{i})\) equals \(1\) if firefighter team \(i\) has been placed on cell \(x\) and \(0\) otherwise.

The reward \(R(x)\) is always negative and is the cost of the fire on cell \(x\).
The reward at time \(t\) is therefore \(\Sigma_{x \in X} B_{t}(x)R(x)\).
The score of a final state is the sum of all rewards.
