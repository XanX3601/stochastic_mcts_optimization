\section{Adapt the \acrshort{nrpa} to stochastic problem}%
\label{sec:adapt_the_nrpa_to_stochastic_problem}

As explained in section \ref{sec:the_nrpa}, the \gls{nrpa} is an algorithm that obtains state of the art results on different optimization problem.
All these problems have in common to be non stochastic problem.
It means that if we apply action \(a\) on state \(s\) and we obtain the state \(s'\) then, when repeating the same process, we will always obtain \(s'\).
In a stochastic context, this assurance does not exist.
Meaning that, when applying \(a\) on \(s\) we may obtain \(s'\) but also \(s^{*}\) or \(s''\)  or any others, depending on the problem.

In this section, we are going to present how we tried to adapt the \gls{nrpa} to solve stochastic problem.
We will first discuss a bit more what a stochastic problem is by considering a small example in order to highlight the difficulty to solve one.
Then we will explain why the \gls{nrpa} is not well suited to solve this kind of problem by identifying the main issues.
Then, we will present how we attended to these issues by explaining our new algorithm the \gls{snrpa}.

\subimport{./subs/}{stochastic.tex}
\subimport{./subs/}{nrpa.tex}
\subimport{./subs/}{snrpa.tex}


