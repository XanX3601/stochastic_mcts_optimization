\subsection{Stochastic problem and how to solve them}%
\label{sub:stochastic_problem_and_how_to_solve_them}

A stochastic problem, modelized as a \gls{mdp} (see section~\ref{sub:mdp}) is a \gls{mdp} in which \(P_{a}(s, s') \leq 1\).
Meaning that the transition between two states \(s\) and \(s'\) by action \(a\) is not guaranteed to succeed.
Therefore, applying any action to any state may, each time, results in different state.

To be clearer, let's consider a small problem.
The rules are simple:

\begin{itemize}
    \item The game is played in a grid of dimension \(W \times H\).
    \item The player starts on a random cell of the grid.
    \item Each turn the player can go \textit{up} , \textit{right} , \textit{down} or \textit{left}.
    \item When choosing a direction, the player may walk one, two or three cells in that direction with a unknown probability.
    \item The player can not exit the grid and is stuck at the edge if he tries.
    \item Each turn, the player freeze a bit and gain \(-1\) point.
    \item An exit is placed on a random cell of the grid.
    \item The goal is for the player to start a turn on the exit.
\end{itemize}

Take the time to understand these rules, they are not very complex.
A state of the game can be represented by the coordinates of the player.
It is a stochastic or random game because, when applying any action to any states, we do not know which state we can obtain.
Exception due to states where the player is on the edges of the grid and try to go beyond.
Now imagine a state where the player is on the next next to the exit.
If he chooses to go toward the exit he might go beyond it.

The difficulty of a stochastic problem is that a sequence \(S\) of \(n\) actions may not contain enough action to solve the game.
Furthermore, many stochastic problems, once modelized as a \gls{mdp} tends to posses a very large set of states.
For some of them, if we were to see the entire tree of the game, the ramification may be too large.
That is not say that for some non stochastic problem, the number of states is not immensely large.
In the game of Go, which is not stochastic, the number of state is \(3^{361}\) which is enormous.
Some problem even have an infinity amount of states.

So the real problem is really the fact that sequence, when applied to a state may not always lead to terminal state.
Also, in some problems, actions can be illegal so a sequence \(S={a1, a2}\) when applied from state \(s\) might lead to a final state \(s^{*}\) but also to another state for which th\(a2\) is illegal.
This poses a new problem as to how define a sequence in a stochastic context.
For some problem, sequences that always lead from root state to a terminal one exists but, in general, it is unlikely.
Furthermore, we are more interested in solving more difficult problem.

